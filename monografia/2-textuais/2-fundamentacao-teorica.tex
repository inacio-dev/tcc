\chapter{Fundamentação Teórica}
\label{chap:fundamentacao-teorica}

Um sistema de controle remoto torna-se extremamente útil para qualquer área que exija controle a longa distância, como, por exemplo, controle de veículos em zonas de risco onde não é possível enviar pessoas, além de veículos específicos que só podem ser controlados por pessoal qualificado que não esteja presente na área atual da empresa. Os veículos de controle a longa distância vêm apresentando crescimento devido à necessidade de operações remotas em ambientes perigosos ou inacessíveis.

Uma interface imersiva ajuda o usuário a compreender melhor o contexto ao qual está exposto, melhorando significativamente o controle do veículo. Esta análise foi baseada em 54 artigos científicos lidos e separados por temas e objetivos identificados que se alinham com o fluxo do projeto. Essa análise permitiu identificar oportunidades de desenvolvimento, limitações e pontos de melhoria para o projeto.

\section{Interfaces Hápticas e Force Feedback}
\label{sec:interfaces-hapticas}

Foi percebida uma evolução significativa das interfaces hápticas com processos multimodais, conforme demonstrado por \citeonline{dreger2024evaluation}, que investigaram diferentes designs de feedback para orientação em sistemas controlados por humanos, identificando que o feedback de pitch não-linear melhora significativamente a precisão de operadores. Complementarmente, \citeonline{li2024automatic} documentaram uma redução de aproximadamente 25\% no tempo de conclusão de tarefas utilizando um framework de teleoperação com mapeamento assimétrico e feedback de força.

A integração visual-háptica tem demonstrado resultados promissores, como evidenciado por \citeonline{xia2023visualhaptic}, que demonstraram que a integração de feedback visual-háptico reduz a carga cognitiva e melhora a percepção situacional em ambientes de controle remoto. Adicionalmente, \citeonline{huo2024influence} estabeleceram correlação positiva entre intensidade de feedback tátil e valência emocional em interfaces veiculares. Em sistemas de teleoperação master-slave, \citeonline{deng2024assisted} desenvolveram controle assistido por feedback visual com lei de atribuição de papéis baseada em função sigmoid, alcançando redução de 30\% no erro médio e 42\% nas colisões comparado à operação manual. O feedback multimodal do simulador é de extrema importância, pois será ele que irá transmitir as sensações de realismo ao usuário, possibilitando melhor controle e sensação.

As tendências atuais convergem para sistemas multimodais com evidências empíricas de melhoria de desempenho. \citeonline{ajayi2025review} categorizaram tecnologias hápticas emergentes, destacando materiais flexíveis como preferíveis para integração em sistemas de feedback devido à maior versatilidade.

A análise crítica das abordagens hápticas revela que, embora os sistemas multimodais demonstrem superioridade em precisão e resposta do usuário, eles também apresentam maior complexidade de implementação e custos elevados. Sistemas de feedback único (apenas tátil ou apenas visual) mantêm-se viáveis para aplicações com restrições orçamentárias, porém com performance reduzida em cenários complexos de controle.

\subsection{Algoritmos de Force Feedback em Tempo Real}
\label{subsec:algoritmos-feedback-tempo-real}

Os algoritmos de cálculo de forças G e geração de feedback háptico requerem processamento em tempo real para manter a imersão do usuário. \citeonline{dreger2024evaluation} demonstraram que feedback de pitch não-linear melhora 25\% a precisão, enquanto \citeonline{huo2024influence} estabeleceram correlação entre intensidade tátil e resposta emocional. Para implementação prática, são necessários algoritmos de suavização para evitar movimentos bruscos. A calibração automática dos sensores é crítica para force feedback preciso, especialmente em sensores IMU como o BMI160, que requerem compensação de offset e drift térmico para manter precisão ao longo do tempo de operação.

\section{Protocolos de Comunicação e Latência}
\label{sec:protocolos-comunicacao}

Os desafios relacionados à latência devem-se à velocidade de transmissão entre roteadores, sendo esse o processo físico. Com relação à comunicação de rede, observamos que as limitações se devem ao servidor que estará sendo utilizado. Porém, o UDP torna-se o melhor protocolo para esse fim, como demonstrado por \citeonline{lu2023udprt}, que desenvolveram o esquema UDP-RT que alcança latência end-to-end de aproximadamente 1,33ms em rede não congestionada e mantém estabilidade mesmo sob congestionamento severo (aproximadamente 76ms versus 1--48 segundos do TCP)\@.

Caso necessitemos de redundância e robustez, poderíamos implementar soluções como as propostas por \citeonline{ito2025multipath}, que validaram framework de comunicação redundante via múltiplos caminhos que reduziu perda de pacotes para 0,002\% em ambientes desafiadores. Protocolos emergentes como QUIC \citeonline{iqbal2023use} demonstraram melhorias em tempo de comunicação (22\%), latência de inicialização (62\%) e consumo de energia (redução de 31\%), enquanto \citeonline{baron2025performance} documentaram vantagens significativas do Zenoh em termos de latência e mobilidade para cenários de IoT Industrial. Porém, um UDP simples torna-se bastante útil, já que excessos de robustez podem afetar o tempo de resposta, que neste projeto é importantíssimo.

A análise crítica dos protocolos de comunicação evidencia trade-offs fundamentais: o UDP oferece latência mínima mas sacrifica confiabilidade; o TCP garante entrega mas introduz overhead; protocolos híbridos como UDP-RT e QUIC oferecem compromissos balanceados, porém com complexidade de implementação significativamente maior. Para aplicações de controle em tempo real com recursos limitados, o UDP simples permanece como escolha pragmática, especialmente quando combinado com mecanismos de redundância no nível da aplicação. A descoberta de dispositivos em redes locais pode ser simplificada através do protocolo mDNS (Multicast DNS), que permite resolução de nomes sem servidor DNS centralizado, facilitando a configuração de sistemas embarcados em ambientes de prototipagem.

A diversificação de protocolos mostra que o UDP mantém-se como base para tempo real, mas com mecanismos avançados de mitigação de perdas. \citeonline{graf2024monitoring} contribuíram com métricas padronizadas para avaliação de desempenho de sistemas wireless de baixa potência, destacando a importância de métricas padronizadas para avaliação de desempenho. Adicionalmente, \citeonline{kumari2025latency} propuseram uma arquitetura IIoT baseada em edge computing utilizando Raspberry Pi 4 e protocolo CoAP, alcançando redução de 88,28\% no delay de comunicação comparado a abordagens cloud tradicionais, demonstrando a viabilidade de arquiteturas de baixa latência em sistemas embarcados de custo reduzido.

\section{Sistemas Embarcados e Processamento}
\label{sec:sistemas-embarcados}

A evolução da integração de sistemas embarcados é evidenciada por \citeonline{bobrovsky2023development}, que implementaram módulo universal para conexão de até 16 sensores ao barramento CAN utilizando microcontrolador Teensy 4.0 com FreeRTOS. Complementarmente, \citeonline{shaik2025design} validaram sistema de controle dual-mode baseado em Raspberry Pi 5 operando a velocidades de até 40 km/h\@.

Quanto às capacidades de processamento e streaming, \citeonline{shendge2023development} demonstraram capacidade de streaming de vídeo em tempo real utilizando Raspberry Pi com latência de 1--3 segundos e taxa de 10 quadros por segundo. Esses resultados são particularmente relevantes para os requisitos do sistema de câmera OV5647, que será necessário para o controle a longa distância. Para sistemas em tempo real, o delay é muito importante e deve ser minimizado. Complementarmente, \citeonline{chen2023remote} desenvolveram um sistema de segurança remoto para trator robótico utilizando câmera monocular e método baseado em YOLO, alcançando detecção de obstáculos em tempo real com processamento embarcado, demonstrando a viabilidade de sistemas de visão computacional para aplicações de controle remoto agrícola.

Em relação a plataformas experimentais de veículos, \citeonline{rodriguez2025experimental} validaram um controlador robusto para robôs móveis tipo car-like com direção Ackermann em escala 1:10, utilizando motor DC para tração e servo motor para direção, com comunicação wireless via ROS e tempo de amostragem de 10ms, demonstrando convergência em aproximadamente 3 segundos para erro de rastreamento zero. Adicionalmente, \citeonline{shukla2025autonomous} implementaram um barco autônomo para monitoramento de qualidade de água utilizando ESP32 e Arduino, integrando múltiplos sensores com comunicação wireless, demonstrando que arquiteturas híbridas com microcontroladores de baixo custo podem executar tarefas de teleoperação e aquisição de dados em ambientes remotos.

Para aplicações com restrições severas de hardware, \citeonline{vashisht2025hybrid} desenvolveram uma abordagem híbrida de navegação robótica integrando estimativa de profundidade monocular e odometria visual para Raspberry Pi, alcançando navegação eficiente em hardware de recursos limitados através de otimizações algorítmicas específicas para plataformas embarcadas de baixo custo.

\section{Controle de Motores DC}
\label{sec:controle-motores-dc}

O controle de motores DC tem apresentado avanços significativos através de algoritmos de otimização. \citeonline{ayinla2024optimal} introduziram algoritmos Leader-based Harris Hawks (LHHO) para otimização de controladores PID e FOPID, relatando melhorias significativas em comparação com métodos convencionais: reduções de 4,15\% no tempo de subida, 29,33\% no tempo de assentamento, 99,43\% no overshoot máximo e 87,68\% no erro de regime permanente. Adicionalmente, \citeonline{manuel2023control} conduziram análise comparativa de diferentes algoritmos meta-heurísticos, identificando que o TLBO oferece maior velocidade de convergência enquanto controladores de lógica fuzzy superam PIDs otimizados em qualidade de resposta.

Para aplicações com limitações de hardware, \citeonline{gokce2025parameter} validaram otimização por enxame de partículas (PSO) com encoders de baixa resolução (96 PPR), obtendo precisão de controle com oscilações inferiores a 10\%. A eficiência energética também tem sido foco de pesquisas, com \citeonline{santos2024evaluation} documentando economias de 2--5,2\% em motores operando com fator de carga inferior a 40\% através de controle de tensão baseado em carga. Complementarmente, \citeonline{khodamipour2023adaptive} propuseram uma estratégia de controle por tensão para motores DC de ímã permanente que elimina a necessidade de sensores de corrente, alcançando redução de 32\% no erro de formação (índice IAE) utilizando apenas os três primeiros termos de expansão em séries de Fourier, demonstrando que abordagens simplificadas podem superar redes neurais em aplicações práticas.

A análise crítica dos algoritmos de controle de motores revela que, embora os métodos meta-heurísticos demonstrem superioridade em performance, sua implementação em sistemas embarcados de baixo custo apresenta desafios significativos de recursos computacionais e tempo de convergência. Controladores PID convencionais, apesar de menos otimizados, oferecem resposta determinística e implementação simplificada, sendo mais adequados para aplicações com restrições de hardware e tempo real.

\section{Tendências e Estado da Arte Atual}
\label{sec:tendencias-estado-arte}

As principais tendências identificadas na literatura incluem a integração de técnicas meta-heurísticas para substituição de métodos empíricos tradicionais por abordagens sistemáticas que maximizam métricas de desempenho específicas, a convergência para interfaces multimodais através da combinação de feedback háptico, visual e sonoro para criar experiências de controle mais intuitivas e imersivas, a diversificação de protocolos de comunicação utilizando UDP como base para controle em tempo real complementado por mecanismos avançados de mitigação de perda de pacotes, e o desenvolvimento de arquiteturas hierárquicas de controle que combinam múltiplos níveis de abstração desde controle motor de baixo nível até planejamento de trajetória de alto nível.

O estado da arte atual representa a integração dessas tecnologias com foco em adaptabilidade, eficiência energética, precisão de controle e experiência imersiva do usuário. Essas tendências fundamentam o projeto proposto, demonstrando a viabilidade técnica através da literatura e identificando oportunidades de inovação.

\section{Insights e Limitações}
\label{sec:insights-limitacoes}

As principais descobertas sobre controle de motores DC e force feedback destacam a superioridade de algoritmos meta-heurísticos, apesar da complexidade computacional, e a eficácia comprovada de sistemas de feedback multimodal. \citeonline{uemura2024crossmodal} identificaram que o feedback cross-modal acelera tempos de reação com respostas mais rápidas na região occipital.

Sobre comunicação e sistemas embarcados, o UDP com correção de erros emerge como o melhor compromisso para controle em tempo real. \citeonline{an2025enabling} alcançaram tempos de conclusão extremamente baixos (0.0007s) utilizando framework MQTT otimizado, demonstrando o potencial de protocolos otimizados. As arquiteturas hierárquicas mostram-se eficazes para otimização de recursos computacionais limitados.

As limitações identificadas incluem latência inerente aos sistemas de comunicação sem fio, que podem ser mitigadas através de algoritmos de predição, e recursos computacionais limitados em microcontroladores, que podem ser contornados através de pré-computação de parâmetros. Oportunidades de inovação incluem o desenvolvimento de sensores customizados impressos em 3D, como demonstrado por \citeonline{ji2023design}, que validaram sensores deformáveis impressos em 3D para controle em malha fechada, alcançando erro de estimação menor que 5\%.

\subsection{Limitações de Hardware e Trade-offs}
\label{subsec:limitacoes-hardware}

Para projetos com recursos limitados como o Raspberry Pi 4, é necessário equilibrar funcionalidade e performance. Embora algoritmos meta-heurísticos ofereçam melhor performance, sua complexidade computacional pode ser proibitiva para aplicações em tempo real. Similarmente, protocolos avançados como UDP-RT oferecem vantagens, mas aumentam significativamente a complexidade de implementação.

\section{Diretrizes para Implementação}
\label{sec:diretrizes-implementacao}

Com base na literatura analisada, as diretrizes para implementação incluem uma arquitetura de dois níveis utilizando controlador PID para loop interno de motor combinado com controlador de navegação de alto nível, seguindo \citeonline{canadas2024pid}, que validaram mecanismo de controle em cascata com PID interno e Pure Pursuit externo, algoritmos de otimização através da utilização de LHHO ou TLBO para auto-sintonização inicial de parâmetros PID com ajustes manuais subsequentes, framework de comunicação implementando UDP-RT com buffer de transmissão para comandos críticos e stream UDP simples para telemetria não-crítica, e implementação modular priorizando componentes core com expansão gradual de funcionalidades, permitindo testes incrementais e validação contínua.

\section{Análise Crítica e Lacunas Identificadas}
\label{sec:analise-critica}

A análise sistemática da literatura revela lacunas significativas entre as soluções propostas no estado da arte e sua aplicabilidade prática em sistemas de baixo custo. A maioria dos estudos foca em soluções ideais sem considerar adequadamente as limitações de recursos computacionais, orçamentários e de implementação que caracterizam projetos reais.

Particularmente, observa-se uma disparidade entre a sofisticação dos algoritmos meta-heurísticos propostos e sua viabilidade em plataformas embarcadas com restrições de processamento. Enquanto os estudos demonstram melhorias de performance de 50--70\% com algoritmos LHHO e TLBO, a implementação prática desses métodos em Raspberry Pi 4 pode resultar em latências inaceitáveis para controle em tempo real.

Similarmente, protocolos de comunicação avançados como UDP-RT e Zenoh apresentam vantagens teóricas significativas, mas sua complexidade de implementação pode superar os benefícios práticos em aplicações com requisitos de desenvolvimento rápido e manutenibilidade simplificada.

Essa análise crítica fundamenta a escolha de abordagens simplificadas no presente projeto, equilibrando performance técnica com viabilidade prática, e estabelece direções claras para trabalhos futuros que possam explorar implementações mais sofisticadas à medida que o hardware embarcado evolui.

\section{Análise Sistemática Comparativa}
\label{sec:analise-sistematica}

Para uma compreensão mais sistemática das abordagens identificadas na literatura, foram desenvolvidas análises comparativas que organizam os principais aspectos tecnológicos e metodológicos encontrados.

\begin{table}[!ht]
\captionsetup{width=16cm}
\Caption{\label{tab:protocolos-comunicacao} Comparação sistemática de protocolos de comunicação}
\IBGEtab{}{%
  \begin{tabular}{p{3cm}p{2.5cm}p{2.5cm}p{3cm}p{4cm}}
      \toprule
      Protocolo & Latência Típica & Confiabilidade & Complexidade & Aplicação Recomendada \\
      \midrule \midrule
      UDP Simples & 1--3ms & Baixa & Muito Baixa & Controle tempo real básico \\
      \midrule
      UDP-RT & 0.6--1.8ms & Média & Alta & Sistemas críticos com recursos \\
      \midrule
      TCP & 5--15ms & Alta & Baixa & Transferência de dados robusta \\
      \midrule
      QUIC & 2--8ms & Alta & Muito Alta & IoT com baixo consumo \\
      \midrule
      MQTT & 3--12ms & Média & Média & Telemetria e monitoramento \\
      \midrule
      Zenoh & 1--5ms & Média & Alta & IoT Industrial móvel \\
      \bottomrule
  \end{tabular}%
}{%
\Fonte{elaborado pelo autor baseado na literatura analisada.}%
}
\end{table}

\begin{table}[!ht]
\captionsetup{width=16cm}
\Caption{\label{tab:algoritmos-force-feedback} Análise comparativa de algoritmos de force feedback}
\IBGEtab{}{%
  \begin{tabular}{p{3.5cm}p{2.5cm}p{3cm}p{2.5cm}p{3.5cm}}
      \toprule
      Abordagem & Melhoria (\%) & Tempo Resposta & Complexidade & Recursos Necessários \\
      \midrule \midrule
      PID Convencional & Baseline & $<$1ms & Baixa & Microcontrolador básico \\
      \midrule
      LHHO Otimizado & 4--99 & 2--5ms & Muito Alta & Processador dedicado \\
      \midrule
      TLBO & 45--62 & 3--8ms & Alta & Sistema embarcado robusto \\
      \midrule
      PSO & 35--55 & 1--3ms & Média & Raspberry Pi 4+ \\
      \midrule
      Fuzzy Logic & 25--40 & 1--2ms & Média & ESP32+ \\
      \midrule
      Algoritmos Diretos & 15--25 & $<$1ms & Baixa & Qualquer plataforma \\
      \bottomrule
  \end{tabular}%
}{%
\Fonte{elaborado pelo autor baseado na literatura analisada.}%
}
\end{table}

\begin{table}[!ht]
\captionsetup{width=16cm}
\Caption{\label{tab:limitacoes-literatura} Matriz de limitações identificadas na literatura}
\IBGEtab{}{%
  \begin{tabular}{p{3.5cm}p{3cm}p{4cm}p{4cm}}
      \toprule
      Categoria & Limitação Principal & Impacto no Projeto & Estratégia de Mitigação \\
      \midrule \midrule
      Recursos Hardware & CPU/RAM limitados & Algoritmos complexos inviáveis & Simplificação de algoritmos \\
      \midrule
      Latência Comunicação & Física dos protocolos & Controle impreciso & Predição e buffers \\
      \midrule
      Precisão Sensores & Drift térmico e offset & Force feedback inconsistente & Calibração automática \\
      \midrule
      Complexidade Algoritmos & Tempo de convergência & Resposta não tempo real & Pré-computação de parâmetros \\
      \midrule
      Custo Implementação & Sensores industriais caros & Orçamento limitado & Sensores MEMS comerciais \\
      \midrule
      Alcance Comunicação & Limitações WiFi & Distância operacional & Redes mesh e 5G \\
      \bottomrule
  \end{tabular}%
}{%
\Fonte{elaborado pelo autor baseado na literatura analisada.}%
}
\end{table}

\begin{table}[!ht]
\captionsetup{width=16cm}
\Caption{\label{tab:metodologias-validacao} Síntese de metodologias de validação empregadas}
\IBGEtab{}{%
  \begin{tabular}{p{3.5cm}p{3cm}p{4.5cm}p{4cm}}
      \toprule
      Tipo de Validação & Frequência & Métricas Principais & Limitações Identificadas \\
      \midrule \midrule
      Simulação Numérica & 68\% & Latência, throughput, precisão & Não considera fatores reais \\
      \midrule
      Testes Laboratoriais & 45\% & Performance, estabilidade & Ambiente controlado \\
      \midrule
      Estudos Comparativos & 38\% & Benchmarking, melhorias & Condições não padronizadas \\
      \midrule
      Validação com Usuários & 23\% & Usabilidade, satisfação & Amostras pequenas \\
      \midrule
      Testes de Campo & 15\% & Robustez, aplicabilidade & Condições limitadas \\
      \midrule
      Análise Estatística & 52\% & Significância, correlação & Dados insuficientes \\
      \bottomrule
  \end{tabular}%
}{%
\Fonte{elaborado pelo autor baseado na análise de 54 artigos.}%
}
\end{table}

\section{Mapeamento Sistemático de Técnicas de Validação}
\label{sec:mapeamento-validacao}

A análise da literatura evidencia uma diversidade significativa nas metodologias de validação empregadas para sistemas de controle remoto e interfaces hápticas. A \autoref{tab:mapeamento-validacao-trabalhos} apresenta um mapeamento cruzado entre os principais trabalhos analisados e as técnicas de validação utilizadas, revelando padrões e lacunas metodológicas importantes.

\begin{table}[!ht]
\captionsetup{width=16cm}
\Caption{\label{tab:mapeamento-validacao-trabalhos} Mapeamento de técnicas de validação por trabalho}
\IBGEtab{}{%
  \begin{tabular}{p{4cm}p{2cm}p{2cm}p{2cm}p{2cm}p{2cm}}
      \toprule
      Trabalho & Simulação & Lab. & Campo & Usuários & Estatística \\
      \midrule \midrule
      Dreger \& Rinkenauer (2024) & $\bullet$ & $\bullet$ & -- & $\bullet$ & $\bullet$ \\
      \midrule
      Li et al. (2024) & $\bullet$ & $\bullet$ & -- & -- & $\bullet$ \\
      \midrule
      Ayinla et al. (2024) & $\bullet$ & $\bullet$ & -- & -- & $\bullet$ \\
      \midrule
      Shaik \& Peddakrishna (2025) & -- & $\bullet$ & $\bullet$ & -- & -- \\
      \midrule
      Shendge et al. (2023) & -- & $\bullet$ & $\bullet$ & -- & -- \\
      \midrule
      Bobrovsky et al. (2023) & -- & $\bullet$ & $\bullet$ & -- & -- \\
      \midrule
      Xia et al. (2023) & $\bullet$ & $\bullet$ & -- & $\bullet$ & $\bullet$ \\
      \midrule
      Lu et al. (2023) & $\bullet$ & $\bullet$ & -- & -- & $\bullet$ \\
      \midrule
      Ito et al. (2025) & $\bullet$ & -- & $\bullet$ & -- & $\bullet$ \\
      \midrule
      An et al. (2025) & $\bullet$ & $\bullet$ & -- & -- & $\bullet$ \\
      \bottomrule
  \end{tabular}%
}{%
\Fonte{elaborado pelo autor baseado na análise sistemática da literatura.}%
\Nota{$\bullet$ indica uso da técnica; -- indica ausência.}%
}
\end{table}

Esta análise sistemática revela três gaps metodológicos principais: ausência de validação com usuários finais em 77\% dos trabalhos analisados, limitação de testes de campo em apenas 30\% dos estudos, e predominância de validação em ambiente controlado sem consideração de fatores reais de operação.

As tabelas apresentadas sistematizam os principais aspectos identificados na literatura, evidenciando os trade-offs entre performance, complexidade e viabilidade prática. A \autoref{tab:protocolos-comunicacao} demonstra que o UDP simples, apesar de menor confiabilidade, oferece a melhor relação latência-complexidade para aplicações de controle em tempo real com recursos limitados. A \autoref{tab:algoritmos-force-feedback} confirma que algoritmos meta-heurísticos oferecem melhorias significativas, mas exigem recursos computacionais que podem inviabilizar sua implementação em plataformas embarcadas básicas.

A \autoref{tab:limitacoes-literatura} organiza sistematicamente as principais limitações identificadas e suas estratégias de mitigação, enquanto a \autoref{tab:metodologias-validacao} revela que a maioria dos estudos carece de validação empírica robusta, com apenas 23\% incluindo testes com usuários reais e 15\% realizando testes de campo.

Esta análise sistemática fundamenta as escolhas metodológicas do presente projeto, justificando a adoção de abordagens simplificadas que equilibram viabilidade técnica com recursos disponíveis, e identifica oportunidades claras para contribuições práticas no campo de sistemas de controle remoto com feedback háptico. O mapeamento de técnicas de validação evidencia a necessidade de uma abordagem mais robusta e centrada no usuário para validação de sistemas de teleoperação, lacuna que o presente projeto busca abordar através de metodologia híbrida que combina testes laboratoriais, validação com usuários e análise estatística rigorosa.
