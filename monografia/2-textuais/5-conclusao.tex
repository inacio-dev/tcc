\chapter{Conclusões e Trabalhos Futuros}
\label{chap:conclusoes-e-trabalhos-futuros}

Foi executado com sucesso a criação de um simulador completo de Fórmula 1 com force feedback e comunicação UDP, visando um controle a longas distâncias e também o desempenho para tempo real e processamento de dados. A utilização de UDP simples foi muito válida para essa situação e o Raspberry Pi 4 foi suficiente para o processamento dos dados do carrinho. O sistema torna-se viável comercialmente já que seu valor foi baixo em relação ao objetivo abrangente e amplo.

\section{Principais Resultados Alcançados}

O sistema desenvolvido apresentou performance superior ao estado da arte, alcançando $3\times$ mais FPS que \citeonline{shendge2023development} (29.9 vs 10), latência $2{,}6\times$ melhor que targets típicos (1.94ms vs 5ms) e resolução $4\times$ maior ($640\times480$ vs $320\times240$). A qualidade de comunicação mostrou-se excelente, com packet loss inferior a 0.4\% (muito abaixo do threshold de 1\%), latência média de 1.94ms ± 0.41ms e throughput estável de 46MB/min. O sistema de force feedback demonstrou eficácia através da detecção de 11.000 eventos de condução, precisão de 97.2\% no cálculo de forças G e execução de 2.713 comandos PWM. A estabilidade operacional foi comprovada com CPU médio de 31\% (operação normal), temperatura controlada entre 45--74°C e FPS estável com 97.1\% de estabilidade.

\section{Limitações Identificadas}

As limitações identificadas relacionam-se principalmente ao servidor UDP utilizado, à latência da rede e também ao alcance entre as antenas WiFi. A utilização em redes Mesh/5G torna o projeto amplamente viável para grandes áreas. Com relação ao force feedback, tudo se resume à força dos atuadores, além da eficiência da bateria do carrinho. A interface gráfica atual é limitada visando apenas o funcional, não profissional, o que torna um excelente ponto de desenvolvimento no futuro.

Os testes realizados foram em ambientes controlados com validações feitas por vários usuários, mas necessitam de testes para condições climáticas adversas, e não temos comparações com simuladores comerciais. A limitação com relação à escala do carrinho afeta as informações hápticas detectadas, mas ainda assim temos uma boa noção para gerar o force feedback.

\section{Trabalhos Futuros}

Para melhorias de curto prazo (3--6 meses), propõe-se o desenvolvimento de interface gráfica profissional utilizando Qt/GTK para melhor experiência do usuário, dashboard com telemetria em tempo real e configurações avançadas de calibração. A otimização de algoritmos incluirá a implementação de algoritmos meta-heurísticos como LHHO e TLBO, calibração automática adaptativa e machine learning para personalização do force feedback. As melhorias de hardware contemplam upgrade para Raspberry Pi 5 visando maior performance, câmera de resolução superior (1080p/60fps) e atuadores mais potentes para force feedback.

Os desenvolvimentos de médio prazo (6--12 meses) incluem comunicação avançada através da implementação de UDP-RT ou QUIC, suporte a 5G para maior alcance e múltiplos clientes simultâneos. Sensores adicionais como sensores impressos 3D customizados, sensores de força nos pneus e telemetria de bateria avançada serão incorporados. A inteligência artificial será aplicada através de piloto automático básico, análise preditiva de telemetria e assistência de condução adaptativa.

Para pesquisa de longo prazo (1--2 anos), a simulação de física avançada contemplará modelagem aerodinâmica realista, simulação de diferentes condições de pista e física de pneus e suspensão. A realidade virtual e aumentada será explorada através da integração com headsets VR, overlay de informações em AR e imersão completa 360°. As aplicações comerciais incluirão versão educacional para escolas, kit DIY para entusiastas e plataforma de competições online.