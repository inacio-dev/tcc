\chapter{Diagrama Elétrico do Raspberry Pi}
\label{apendice:diagrama-rpi}

Este apêndice apresenta o diagrama elétrico completo das conexões do Raspberry Pi 4 com os demais componentes do sistema: sensor BMI160, driver PCA9685, sensor de corrente INA219, ponte H BTS7960, Arduino Pro Micro para monitoramento de energia e demais periféricos.

\section*{Pinagem do Raspberry Pi 4}

A \autoref{tab:pinagem-rpi} apresenta a pinagem completa do Raspberry Pi 4 utilizada no veículo teleoperado.

\begin{table}[!h]
\captionsetup{width=16cm}
\Caption{\label{tab:pinagem-rpi} Mapeamento de pinos GPIO do Raspberry Pi 4 no veículo}
\IBGEtab{}{%
	\begin{tabular}{cccp{6cm}}
		\toprule
		GPIO & Componente & Sinal & Observação \\
		\midrule \midrule
		2 & Barramento I2C & SDA & Compartilhado: BMI160, PCA9685, INA219 \\
		\midrule
		3 & Barramento I2C & SCL & Compartilhado: BMI160, PCA9685, INA219 \\
		\midrule
		4 & DS18B20 & DATA & 1-Wire com pull-up 4,7k$\Omega$ \\
		\midrule
		18 & BTS7960 & RPWM & PWM motor horário \\
		\midrule
		27 & BTS7960 & LPWM & PWM motor anti-horário \\
		\midrule
		22 & BTS7960 & R\_EN & Enable lado direito \\
		\midrule
		23 & BTS7960 & L\_EN & Enable lado esquerdo \\
		\midrule
		CSI & Câmera OV5647 & Flat cable & Interface Camera Serial Interface \\
		\midrule
		USB & Arduino Pro Micro & Serial & 115200 baud, monitoramento de energia \\
		\bottomrule
	\end{tabular}%
}{%
\Fonte{elaborado pelo autor.}%
}
\end{table}

A \autoref{tab:enderecos-i2c-rpi} apresenta os endereços I2C dos dispositivos conectados ao barramento compartilhado.

\begin{table}[!h]
\captionsetup{width=16cm}
\Caption{\label{tab:enderecos-i2c-rpi} Endereços I2C dos dispositivos no veículo}
\IBGEtab{}{%
	\begin{tabular}{ccp{8cm}}
		\toprule
		Endereço & Dispositivo & Função \\
		\midrule \midrule
		0x40 & PCA9685 & Driver PWM 16 canais para servos \\
		\midrule
		0x41 & INA219 & Sensor de corrente/tensão do Raspberry Pi (A0=VCC) \\
		\midrule
		0x68 & BMI160 & IMU 6 eixos (acelerômetro/giroscópio) \\
		\bottomrule
	\end{tabular}%
}{%
\Fonte{elaborado pelo autor.}%
}
\end{table}

\section*{Diagrama de Alimentação e Sensores de Corrente}

Diagrama mostrando a bateria LiPo Turnigy Grafeno 6000mAh 3S 150C, regulador step-down XL4015 5A, sensores de corrente ACS758 (50A e 100A) com filtros RC (1k$\Omega$ e 100nF), divisor de tensão (20k$\Omega$/10k$\Omega$), Arduino Pro Micro para aquisição de dados via USB serial, UBEC 15A para alimentação dos servos, driver PWM PCA9685, ponte H BTS7960B e motor DC 775.

\begin{center}
\includegraphics[width=\textwidth,height=0.85\textheight,keepaspectratio]{figuras/diagrama-rpi-1}
\end{center}

\newpage

\section*{Raspberry Pi, Sensores e Comunicação}

Diagrama mostrando o Raspberry Pi 4 Modelo B como elemento central, sensor ACS758 50A conectado ao Arduino Pro Micro via pino A1, driver PWM PCA9685 de 16 canais com pino OE, UBEC 15A, sensor de corrente e tensão INA219 I2C, módulo USB-C breakout com resistores de 5,1k$\Omega$ para negociação de energia, câmera OV5647 conectada via cabo flat CSI, sensor de temperatura DS18B20 com resistor pull-up de 4,7k$\Omega$ e sensor IMU GY-BMI160 conectado via I2C.

\begin{center}
\includegraphics[width=\textwidth,height=0.85\textheight,keepaspectratio]{figuras/diagrama-rpi-2}
\end{center}
