\chapter{Diagrama Elétrico do Raspberry Pi}
\label{apendice:diagrama-rpi}

Este apêndice apresenta o diagrama elétrico completo das conexões do Raspberry Pi 4 com os demais componentes do sistema: sensor BMI160, driver PCA9685, ADC ADS1115, sensor de corrente INA219, ponte H BTS7960 e demais periféricos. O diagrama está dividido em cinco partes para melhor visualização.

\section*{Pinagem do Raspberry Pi 4}

A \autoref{tab:pinagem-rpi} apresenta a pinagem completa do Raspberry Pi 4 utilizada no veículo teleoperado.

\begin{table}[!h]
\captionsetup{width=16cm}
\Caption{\label{tab:pinagem-rpi} Mapeamento de pinos GPIO do Raspberry Pi 4 no veículo}
\IBGEtab{}{%
	\begin{tabular}{cccp{6cm}}
		\toprule
		GPIO & Componente & Sinal & Observação \\
		\midrule \midrule
		2 & Barramento I2C & SDA & Compartilhado: BMI160, PCA9685, ADS1115, INA219 \\
		\midrule
		3 & Barramento I2C & SCL & Compartilhado: BMI160, PCA9685, ADS1115, INA219 \\
		\midrule
		4 & DS18B20 & DATA & 1-Wire com pull-up 4,7k$\Omega$ \\
		\midrule
		18 & BTS7960 & RPWM & PWM motor horário \\
		\midrule
		27 & BTS7960 & LPWM & PWM motor anti-horário \\
		\midrule
		22 & BTS7960 & R\_EN & Enable lado direito \\
		\midrule
		23 & BTS7960 & L\_EN & Enable lado esquerdo \\
		\midrule
		CSI & Câmera OV5647 & Flat cable & Interface Camera Serial Interface \\
		\bottomrule
	\end{tabular}%
}{%
\Fonte{elaborado pelo autor.}%
}
\end{table}

A \autoref{tab:enderecos-i2c-rpi} apresenta os endereços I2C dos dispositivos conectados ao barramento compartilhado.

\begin{table}[!h]
\captionsetup{width=16cm}
\Caption{\label{tab:enderecos-i2c-rpi} Endereços I2C dos dispositivos no veículo}
\IBGEtab{}{%
	\begin{tabular}{ccp{8cm}}
		\toprule
		Endereço & Dispositivo & Função \\
		\midrule \midrule
		0x40 & PCA9685 & Driver PWM 16 canais para servos \\
		\midrule
		0x41 & INA219 & Sensor de corrente/tensão do Raspberry Pi (A0=VCC) \\
		\midrule
		0x48 & ADS1115 & ADC 16 bits para sensores ACS758 \\
		\midrule
		0x68 & BMI160 & IMU 6 eixos (acelerômetro/giroscópio) \\
		\bottomrule
	\end{tabular}%
}{%
\Fonte{elaborado pelo autor.}%
}
\end{table}

\section*{Parte 1 - Alimentação Principal e Sensores de Corrente}

Seção do diagrama mostrando a bateria LiPo Turnigy Grafeno 6000mAh 3S 150C, regulador step-down XL4015 5A, sensores de corrente ACS758 (50A e 100A) com filtros RC (1k$\Omega$ e 100nF), e conexão com a ponte H BTS7960B.

\begin{center}
\includegraphics[width=\textwidth,height=0.8\textheight,keepaspectratio]{figuras/diagrama-rpi-1}
\end{center}

\newpage

\section*{Parte 2 - UBEC, Driver PWM e Motor de Propulsão}

Seção do diagrama mostrando o UBEC 15A para alimentação dos servos, driver PWM PCA9685 de 16 canais conectado ao Raspberry Pi 4 via I2C (GPIO2/GPIO3), sensor de corrente ACS758 50A para monitoramento dos servos, motor DC 775 e ponte H BTS7960B.

\begin{center}
\includegraphics[width=\textwidth,height=0.8\textheight,keepaspectratio]{figuras/diagrama-rpi-2}
\end{center}

\newpage

\section*{Parte 3 - ADC e Monitoramento de Corrente}

Seção do diagrama mostrando a ponte H BTS7960B, sensores de corrente ACS758 (100A para motor e 50A para Raspberry Pi), regulador XL4015, ADC ADS1115 de 16 bits conectado via I2C, e filtros RC para condicionamento dos sinais analógicos.

\begin{center}
\includegraphics[width=\textwidth,height=0.8\textheight,keepaspectratio]{figuras/diagrama-rpi-3}
\end{center}

\newpage

\section*{Parte 4 - Alimentação USB-C e Sensor INA219}

Seção do diagrama mostrando a chave ON/OFF, sensor de corrente e tensão INA219 I2C, módulo USB-C breakout com resistores de 5,1k$\Omega$ para negociação de energia, divisores de tensão com resistores de 10k$\Omega$, e conexão de alimentação com o Raspberry Pi 4.

\begin{center}
\includegraphics[width=\textwidth,height=0.8\textheight,keepaspectratio]{figuras/diagrama-rpi-4}
\end{center}

\newpage

\section*{Parte 5 - Raspberry Pi e Sensores}

Seção do diagrama mostrando o Raspberry Pi 4 Modelo B como elemento central, ADC ADS1115 conectado via I2C, câmera OV5647 conectada via cabo flat CSI, sensor de temperatura DS18B20 com resistor pull-up de 4,7k$\Omega$, sensor IMU GY-BMI160 conectado via I2C, e módulo USB-C breakout.

\begin{center}
\includegraphics[width=\textwidth,height=0.8\textheight,keepaspectratio]{figuras/diagrama-rpi-5}
\end{center}
