\chapter{Diagrama Elétrico do Simulador (ESP32)}
\label{apendice:diagrama-esp32}

Este apêndice apresenta o diagrama elétrico completo das conexões do ESP32 DevKit V1 com os demais componentes do simulador: encoders rotacionais LPD3806-600BM, ponte H BTS7960 para force feedback, botões de troca de marcha, e sistema de alimentação via fonte ATX com placa breakout.

\section*{Pinagem do ESP32 DevKit V1}

A \autoref{tab:pinagem-esp32} apresenta a pinagem completa do ESP32 DevKit V1 utilizada no simulador.

\begin{table}[!h]
\captionsetup{width=16cm}
\Caption{\label{tab:pinagem-esp32} Mapeamento de pinos GPIO do ESP32 no simulador}
\IBGEtab{}{%
	\begin{tabular}{cccp{6cm}}
		\toprule
		GPIO & Componente & Sinal & Observação \\
		\midrule \midrule
		25 & Encoder Acelerador & CLK & Pinos invertidos para correção de direção \\
		\midrule
		26 & Encoder Acelerador & DT & --- \\
		\midrule
		27 & Encoder Freio & CLK & --- \\
		\midrule
		14 & Encoder Freio & DT & --- \\
		\midrule
		12 & Encoder Direção & CLK & Pinos invertidos para correção de direção \\
		\midrule
		13 & Encoder Direção & DT & --- \\
		\midrule
		32 & Botão Marcha & UP & Pull-up interno ativado \\
		\midrule
		33 & Botão Marcha & DOWN & Pull-up interno ativado \\
		\midrule
		16 & BTS7960 & RPWM & PWM force feedback horário \\
		\midrule
		17 & BTS7960 & LPWM & PWM force feedback anti-horário \\
		\midrule
		18 & BTS7960 & R\_EN & Enable lado direito \\
		\midrule
		19 & BTS7960 & L\_EN & Enable lado esquerdo \\
		\bottomrule
	\end{tabular}%
}{%
\Fonte{elaborado pelo autor.}%
}
\end{table}

\section*{Diagrama de Conexões}

% Espaço reservado para imagens do diagrama elétrico do ESP32
% \begin{center}
% \includegraphics[width=\textwidth,height=0.8\textheight,keepaspectratio]{figuras/diagrama-esp32-1}
% \end{center}

