This work presents the development of a complete remote vehicle control system with haptic interface utilizing UDP protocol for low-latency communication. The objective was to design and implement a three-layer architecture comprising: a remotely controlled vehicle based on Raspberry Pi 4 with OV5647 camera and BMI160 sensors; UDP communication optimized for latency below 5ms; and a physical cockpit with force feedback controlled via Arduino Mega integrated with a Python/Tkinter graphical interface. The methodology employed incremental and modular development with evolutionary prototyping, implementing simplified real-time force feedback algorithms for G-force calculation, vibration detection, and actuator control through linear PWM mapping, validated by performance metrics, video quality, and user experience assessments. During a 15-minute experimental session, over 90,000 telemetry data points and 26,925 video frames were collected, achieving an average latency of 1.94ms ± 0.41ms (2.6x better than typical 5ms targets), 29.9 FPS (3x superior to the state-of-the-art 10 FPS), packet loss below 0.28\%, and effective detection of 11,000 force feedback events with 97.2\% accuracy in G-force calculations. The results demonstrate that the simplified approach using plain UDP and direct force feedback algorithms outperforms complex protocols like UDP-RT for low-cost teleoperation applications, validating the technical and commercial viability of the system with a total cost of R\$ 1,300 and energy efficiency of 123.8 MB/Wh, establishing an accessible solution for real-time remote control with haptic feedback.

\keywords{Remote control. Force feedback. UDP protocol. Embedded systems. Raspberry Pi. Haptic interface.}