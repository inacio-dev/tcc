Este trabalho apresenta o desenvolvimento de um sistema completo de controle remoto de veículos com interface háptica utilizando protocolo UDP para comunicação de baixa latência. O objetivo foi projetar e implementar uma arquitetura de três camadas compreendendo: veículo teleoperado baseado em Raspberry Pi 4 com câmera OV5647 e sensores BMI160; cliente em Python/Tkinter para interface gráfica e comunicação UDP otimizada para latência inferior a 5ms; e simulador físico com volante, pedais e force feedback controlado via ESP32. A metodologia empregou desenvolvimento incremental e modular com prototipagem evolutiva, implementando algoritmos simplificados de force feedback em tempo real para cálculo de forças G, detecção de vibrações e controle de atuadores através de mapeamento linear PWM, validados por métricas de performance, qualidade de vídeo e experiência do usuário. Durante sessão experimental de 15 minutos foram coletados mais de 90.000 pontos de telemetria e 26.925 frames de vídeo, alcançando latência média de 1.94ms ± 0.41ms (2.6x melhor que targets típicos de 5ms), FPS de 29.9 (3x superior ao estado da arte de 10 FPS), packet loss inferior a 0.28\%, e detecção eficaz de 11.000 eventos de force feedback com precisão de 97.2\% no cálculo de forças G. Os resultados demonstram que a abordagem simplificada com UDP simples e algoritmos diretos de force feedback supera protocolos complexos como UDP-RT para aplicações de teleoperação de baixo custo, validando a viabilidade técnica e comercial do sistema com custo total de R\$ 1.300 e eficiência energética de 123.8 MB/Wh, estabelecendo uma solução acessível para controle remoto com feedback háptico em tempo real.

\palavraschave{Controle remoto. Force feedback. Protocolo UDP. Sistemas embarcados. Raspberry Pi. Interface háptica.}